Argument je skup premisa i dokaza koje podupiru neki sud. Sud je propozicija ili tvrdnja \engl{claim} kojoj se pridjeljuje istinosna vrijednost (\emph{istina} ili \emph{laž}). Sudovima i zaključivanjem se bavi logika. \citep{vukovic2007matematicka}. Sudovi se donose (dodjeljuje im se istinosna vrijednost) na temelju argumenata i procesa argumentiranja.  

Argumentiranje je verbalna i društvena aktivnost s ciljem jačanja(ili osporavanja) diskutabilnog stajališta. Sredstvo argumentiranja su prijedlozi ili propozicije \engl{propositions} koji opravdavaju ili opovrgavaju stajalište nepristranom sucu s mogućnošću nepristranog racionalnog prosuđivanja \citep{rahwan2006representing}. Argumentiranje se često proučava u sklopu analize diskursa kao specifičan oblik razgovora \citep{palau2009argumentation}. Prema teoriji iz \citep{van2003systematic} argumentiranje je uvijek dio dijaloga u kojem jedna strana pokušava uvjeriti drugu u ispravnost svojih stajališta. Dijalog ili tekst u kojem autor iznosi svoja stajališta s pretpostavkom da očekuje protuargumente, kritike ili sumnje se smatra slobodnim \engl{free text}. 

Teorija argumenta i argumentiranja vrlo je vjerno preslikana u pravnom sustavu. Prilikom odlučivanja o nekom diskutabilnoj temi, suprotstavljene teme iznose svoje argumente vezane uz temu. Iznesene argumente evaluira racionalan i nepristran sudac te donosi odluku kojom se rješava diskutabilnost teme.
Jednostavan primjer primjene argumentiranja u sudstvu bi bio spor oko vlasništva zemlje između susjeda. Svaki od susjeda tvrdi da mu pripada vlasništvo nad istim posjedom. Dakle, sudovi izgledaju:

\begin{itemize}
\item Susjed A tvrdi da mu pripada posjed X.
\item Susjed B tvrdi da mu pripada posjed X.
\end{itemize}

Oba suda mogu biti podržana različtim argumentima:

\begin{itemize}
\item Susjed A tvrdi da mu je posjed X pripadao u zadnjih 10 godina.
\item Susjed B tvrdi da mu je posjed X otkupio o drugog vlasnika od susjeda A.
\end{itemize}

Ispravnost argumentata ocjenjuje sudac (ili porota) nakon čega se oba početna suda interpretiraju (dodjeljuje im se logička vrijednost istine ili laži).  

%sada kad sam definirao što je uopće argument (vjerojatno će ići velika redukcija ovog napisanog), idem objasniti kako pronaći argument u tekstu s primjerima

\section{Definicija argumenta}

Argument može imati različite namjene. Izražavanjem \textbf{suda}, što se naziva i \textbf{izjavom} \engl{statement}, pridjeljujemo istinosnu vrijednost danoj izjavi. 
\vspace{2 mm}

\fbox{
 \addtolength{\linewidth}{-2\fboxsep}%
 \addtolength{\linewidth}{-2\fboxrule}%
 \begin{minipage}{13 cm}
  Primjer: Završio sam diplomski studij na FERu.
 \end{minipage}
} 

\vspace{2 mm}
\textbf{Uvjetni sud} \engl{conditional statement} je posebna vrsta izjave, jer razdvaja argumente na \textbf{antecedent} i \textbf{konsekvens}. Najčešće se pojavljuje kao pogodbena složena rečenica u obliku "Ako [antecedent], onda [konsekvens]".
\vspace{2 mm}

\fbox{
 \addtolength{\linewidth}{-2\fboxsep}%
 \addtolength{\linewidth}{-2\fboxrule}%
 \begin{minipage}{13 cm}
  Primjer: Ako sam završio diplomski studij na FERu, onda mogu upisati poslijediplomski studij.
 \end{minipage}
} 

\vspace{2 mm}
Čitava rečenica iz gore navedenog primjera smatra se uvjetnim sudom gdje je "Ako sam završio diplomski studij na FERu" antecedent, a "onda mogu upisati poslijediplomski studij" je konsekvens. Kao i običnom sudu, uvjetnom sudu moguće je pridjeliti istinosnu vrijednost. 

\textbf{Argument} je skup izjava, od kojih je jedna \emph{zaključak}, a ostale su \emph{premise}. Zaključak se donosi na temelju premisa.
\vspace{2 mm}

\fbox{
 \addtolength{\linewidth}{-2\fboxsep}%
 \addtolength{\linewidth}{-2\fboxrule}%
 \begin{minipage}{13 cm}
Primjer: Prema statutu je dozvoljeno upisati poslijediplomski studij samo s prosjekom ocjena višim od 3.5 tijekom diplomskog studija. Završio sam diplomski studij s prosjekom ocjena višim od 3.5, tako da bih mogao upisati poslijediplomski studij.
 \end{minipage}
}
\vspace{2 mm}

Ovdje navodnimo dvije premise, iz čega se izvodi zaključak. Argument se valorizira s obzirom na to koliko dobro premise podupiru zaključak. U ovom jednostavnom primjeru, zaključak ovisi o dvije premise koje obje podupiru zaključak. 

\section{Prepoznavanje argumenata}

Prepoznavanje argumenata u rečenici smatra se teškim problemom, i to ne samo računalna. Bogatstvo jezičnog izražaja, složeni rečenični konstrukti, implicitno zaključivanje ili uporaba ironije su samo neki od faktora koji mogu značajno otežati pronalazak argumenata u tekstu. U \citep{harrellcreating} su navedene smjernice za izdvajanje argumenata iz nestrukturiranih tekstova. Nestrukturirani tekstovi mogu biti osvrti, blogovi, odgovori na internetskim forumima, sudski zapisnici, političke zakonske rasprave \dots
U nestrukturiranim tekstovima premise i zaključci su međusobno isprepleteni, ponekad \emph{skriveni između redaka} ili povezani u istim rečenicama. 
\vspace{1 mm}

\fbox{
 \addtolength{\linewidth}{-2\fboxsep}%
 \addtolength{\linewidth}{-2\fboxrule}%
 \begin{minipage}{13 cm}
Primjer: Treba ukinuti porez na nekretnine jer neće povećati prihod u proračunu.
 \end{minipage}
}
\vspace{2 mm}

U ovom primjeru su premisa (povećanje prihoda u proračunu) i zaključak (ukidanje poreza na nekretnine) navedeni u istoj rečenici, stoga je i ovo primjer argumenta.

S obzirom na vrste riječi u rečenici argumenti se često vežu uz karakteristične uzročne i posljedične priloge, frazeme te često korištene singtagme. Svaki jezik ima svoje specifične karakteristike, stoga je identifikaciju tih ključnih riječi potrebno provesti za svaki jezik. Riječi koje upućuju da slijedi premisa mogu biti:
\begin{center}
\begin{tabular}{c c c c}
jer & zbog & iz razloga što & kako bi  \\
budući da & zato što & unatoč & pretpostavka da
\end{tabular}
\end{center}

Na isti način moguće je pronaći riječi koje se vežu uz zaključak. Identifikacijom tipičnih riječi za argumente može se olakšati posao prepoznavanja argumenata u tekstu. 
Svaki klasifikacijski zadatak često se mora znati nositi s šumovitim podacima. 
Autor teksta može iznijeti tuđi argument s kojim se ne slaže, ne navodeći razloge (premise) \engl{discount}.

\fbox{
 \addtolength{\linewidth}{-2\fboxsep}%
 \addtolength{\linewidth}{-2\fboxrule}%
 \begin{minipage}{13 cm}
Primjer: Mada poneki događaji na financijskoj burzi dionica djeluju nasumično, prema financijskim stručnjacima i njihovim dugogodišnjim istraživanjima velikih korporacija, možemo reći kako je globalno financijsko stanje stabilno. 
 \end{minipage}
}
\vspace{2 mm}

U ovom slučaju imamo glavnu premisu o istraživanju burze, koja je \emph{lažno} devalorizirana nasumičnim događajima. Moguće je ponoviti istu premisu drukčijim riječima \engl{repetition}, izraziti iznimno visoku \engl{assurance} ili nisku pouzdanost \engl{hedge} nekog događaja u obliku argumenata. 

\fbox{
 \addtolength{\linewidth}{-2\fboxsep}%
 \addtolength{\linewidth}{-2\fboxrule}%
 \begin{minipage}{13 cm}
Primjer: Očito je kako je ova generacija studenata bolja od prethodne, jer postiže mnogo bolje rezultate. No, po mom mišljenju, iduća generacija bi mogla biti još bolja.
 \end{minipage}
}
\vspace{2 mm}

Rečenične konstrukcije koje uključuju frazeme kao što su "očito je", "po mom mišljenju" i pokazuju tendencije autora prema nekom argumentu ili se koristi kako bi se čitatelju sugeriralo kako je autor lišen predrasuda. \citep{harrellcreating}. 

\textbf{Entimem} \engl{enthymeme} je argument gdje premisa i/ili zaključak nije eksplicitan \citep{bitzer1959aristotle}. Entimemi se rabe u slučajevima kada autor smatra da nije potrebno eksplicitno naglasiti zaključak.

\section{Struktura argumenata}
Odnosi između argumenata mogu biti različiti. 

\vspace{1 mm}

\fbox{\addtolength{\linewidth}{-2\fboxsep}
\addtolength{\linewidth}{-2\fboxrule}
 \begin{minipage}{13 cm}
Primjer: Farmaceutska istraživanja otkrivaju nove lijekove za suzbijanje
bolesti, a zdravstvo mora biti prioritet svake državne organizacije. Dakle,
vlada bi trebala izdvajati veća sredstva za istraživanja.
 \end{minipage}
}

U ovom primjeru, razlikujemo dvije premise:
\begin{itemize}
\item farmaceutska istraživanja otkrivaju nove lijekove i 
\item zdravstvo mora biti prioritet svake državne organizacije,
\end{itemize} 
koje podupiru jedinstveni zaključak: 
\begin{itemize}
\item vlada bi trebala izdvajati veća sredstva za istraživanja.
\end{itemize}

Ovo je primjer jednog odnosa argumentiranja. Argumentiranje može biti:
\begin{itemize}
\label{itemize:naciniarg}
\item jednostavno argumentiranje (stajalište se podupire jednom tvrdnjom), 
\item višestruko argumentiranje (stajalište se podupire s višestrukim tvrdnjama) i
\item složeno argumentiranje (tvrdnje se međusobno \emph{ulančavaju}) koje može biti:
	\begin{itemize}
	\item koordinirano složeno argumentiranje (\emph{paralelno} podupiranje stajališta)
	\item podređeno složeno argumentiranje (tvrdnje se \emph{serijski} podupiru). 
	\end{itemize}
\end{itemize}

Proučavanjem dijaloga i argumenata omogućilo je strukturiranje argumenata u
grafove \citep{wolf2006coherence}. 


\subsection{Povijest mapiranja argumenata}

Analiza i prikazivanje argumenata u pravnom sustavu se nameće samo po sebi.
Prvi koji se pokušao poslužiti mapiranjem argumenata kako bi opisao pravni
slučaj bio je John H. Wigmore. Doduše, njegove metode mapiranja nikad nisu
doživile svoju praktičnu primjenu na većem broju slučaja. Richard Whately
(1859) se također bavio analizom argumenta.

Kasnije se strukturiranje argumenata počelo više koristiti u računarstvu,
točnije, obradi prirodnog jezika \citep{reed2003argumentation}. 

\subsection{Argumentacijske sheme}


Dvije najkorištenije argumentacijs



\section{Pretpostavke sustava argumentiranja}
Dokazi i argumenti su dva bitno različita pojma, pa tako izjave: \emph{P je dokaz da vrijedi T} i \emph{P je čvrst argument u korist prihvaćanja teze T} nikako ne mogu biti smatrane istovjetnima. Prva izjava pripada području matematička logike \engl{mathematical reasoning}. U matematičkoj logici \citep{bench2007argumentation}:
\begin{itemize}
\item ne postoje pojmovi nepotpune ili nesigurne informacije,
\item zaključci su konačni,
\item kontekst rasuđivanja je strogo definiran i
\item sustav odlučivanja nije podložan raspravi te je potpuno objektivan. 
\end{itemize} 

Zaključno, cilj argumenta je \textbf{uvjeravanje} \engl{to persuade}.