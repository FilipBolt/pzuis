\textbf{Argument} je sredstvo razrješavanja sukoba oko kontroverzne izjave 
\engl{claim} \citep{walton1990reasoning} između dva ili više sudionika.
Sastoji se od barem jedne pretpostavke ili \textbf{premise} \engl{premise}
i \textbf{zaključka} \engl{conclusion} koji logički slijedi iz pretpostavki kroz
\textbf{postupak zaključivanja} \engl{reasoning}. 
Središnji dio argumenta je zaključak, koji se smatra
kontroveznom izjavom ili tvrdnjom, a cilj iznošenja argumenata je
uvjeriti publiku kako je argument ispravan \citep{walton1990reasoning}. 
Premise (pretpostavke) opravdavaju \engl{justify}  
tvrdnju argumenta~\citep{besnard2008elements}. 
Primjerice, argument $A$ sadrži rečenice: 

\begin{exe}
    \ex\label{ex1} student Ivo je uspješno predao seminar iz predmeta 
    \textit{Predstavljanje znanja u skupovima podataka},
    \ex\label{ex2} svaki student koji uspješno preda seminar iz predmeta 
    \textit{Predstavljanje znanja u skupovima podataka} je položio predmet 
    \textit{Predstavljanje znanja u skupovima podataka}, i
    \ex\label{ex3} student Ivo je položio predmet 
    \textit{Predstavljanje znanja u skupovima podataka}. 
\end{exe} 
gdje su (\ref{ex1}) i (\ref{ex2}) premise argumenta $A$ iz kojih slijedi tvrdnja (zaključak)
(\ref{ex3}).

\cite{toulmin1974uses}, vrlo utjecajni filozof i retoričar 20.\ stoljeća, definirao je 
strukturu kvalitetnog argumenta. 
Specijalizirao je uloge premisa i tvrdnji tako što je definirao: 
\begin{enumerate}
    \item \textbf{tvrdnju} \engl{claim}: središnju izjavu argumenta,
    \item \textbf{podatke} \engl{data}: činjenice i dokaze argumenta,
    \item \textbf{razlog} \engl{warrant}: logičku vezu između tvrdnje i podataka,
    \item \textbf{pozadinu} \engl{backing}: izjave koje podupiru razlog,
    \item \textbf{kvalifikator} \engl{qualifier}: uvjet istinitosti tvrdnje ili snaga tvrdnje, i
    \item \textbf{pobijanje} \engl{rebuttal}: izjave koje ističu uvjete pod kojima argument nije istinit.
\end{enumerate}
Elementi Toulminovog argumenta i njihovi međuodnosi prikazani su na slici~\ref{fig:Toulmin}.  
\begin{figure}
\centering
\begin{tikzpicture}
[node distance = 1cm, auto,font=\footnotesize,
% STYLES
every node/.style={node distance=3cm},
% The comment style is used to describe the characteristics of each force
comment/.style={rectangle, inner sep= 5pt, text width=4cm, node distance=0.25cm, font=\scriptsize\sffamily},
% The force style is used to draw the forces' name
force/.style={rectangle, draw, fill=black!10, inner sep=5pt, text badly centered, minimum height=1.2cm, font=\bfseries\footnotesize\sffamily}] 

% glavni dijelovi argumenta

\node [force] (claim) {Tvrdnja};

\node [force, left=4 cm of claim] (data) {Podatak};

\node [force, below of=data, xshift=2cm] (warrant) {Razlog};

% \node [force, right=2.5cm of claim] (rebuttal) {Pobijanje};

\node [force, below of=claim] (qualifier) {Kvalifikator};

\node [force, below of=warrant] (backing) {Pozadina};

\node [force, right=1.5 cm of claim] (rebuttal) {Pobijanje};

% komentari

\node [comment, above=0cm of claim] (claim-text) {Student Ivo bi trebao položiti PZUIS.};

\node [comment, above=0cm of data] (data-text) {Student Ivo je predao seminar iz PZUIS-a. };

\node [comment, left=0cm of warrant] (warrant-text) {Student Ivo je poslao mailom 
završeni seminarski rad iz PZUIS-a}; 

\node [comment, left=0cm of backing] (backing-text) {Predaja seminara je uvjet za polaganje predmeta. };

% \node [comment, below=0cm of rebuttal] (rebuttal-text) {Student Ivo nije ispunio obveze PZUIS-a na vrijeme. };

\node [comment, below=0cm of qualifier] (qualifier-text) {Većina studenata koja ispuni obveze prođe PZUIS. };

\node [comment, below=0cm of rebuttal] (rebuttal-text) {Student Ivo nije ispravno upisao PZUIS. };

% Draw the links between forces
\path[->,thick] 
 % (rebuttal) edge (claim)
 (data) edge (claim)
 (backing) edge (warrant)
 (qualifier) edge (claim)
 (rebuttal) edge (claim)
 ;

\path [->,draw,thick] 
(warrant) -- ($ (data)!.35! (claim) $);

\end{tikzpicture} 
\caption{Toulminov primjer kvalitetnog argumenta}
\label{fig:Toulmin}
\end{figure}
 Zbog svoje jednostavnosti, ali i 
lake mogućnosti primjene na stvarne argumentativne rasprave, 
Toulminovom strukturom argumenta (Toulminovim modelom) najčešće se
modelira argumentacija u području umjetne inteligencije. 

Argumenti se mogu međusobno povezivati, uspoređivati i vrednovati.  Kažemo da
argument $A$ pobija \engl{rebutts} argument $B$ ukoliko argument $A$ sadrži
tvrdnju koja je kontradiktorna tvrdnji ili bilo kojoj premisi iz argumenta
$B$~\citep{besnard2008elements}. Primjer argumenta $B$ u kontradikciji s
argumentom sa slike~\ref{fig:Toulmin} ima tvrdnju:
\begin{exe}
    \ex\label{exb1} student Ivo nije obavio predaju seminara iz PZUIS-a na vrijeme. 
\end{exe}
gdje je tvrdnja (\ref{exb1}) iz argumenta $B$ u kontradikciji s premisom (\ref{ex1})
argumenta sa slike~\ref{fig:Toulmin}. 

Argumenti se međusobno povezuju postupcima zaključivanja \engl{reasoning}. 
Izvođenje zaključaka ili tvrdnji iz premisa može se raditi pod različitim pravilima 
logike. 
Tako razlikujemo logiku prvog reda, predikatnu logiku, neformalnu i brojne druge. 
Neformalna logika \engl{informal logic} formalizira, 
razvija metode evaluacije i vrednovanja
argumentacije svakodnevnog razgovora~\citep{fogelin1985logic, blair2000informal}. 
Upravo se na neformalnu logiku naslanja moderna teorija argumentacije procjenjuje
valjanost argumenata u raspravama pravilima neformalne logike.
Više o zaključivanju i valjanosti argumenata u poglavlju~\ref{chap:eval}.

\section{Vrste informacija u argumentaciji}

\cite{besnard2008elements}~razlikuju subjektivne, objektivne ili hipotetske informacije u 
argumentaciji. 
S obzirom na izrazitost, informacije mogu biti 
sigurne \engl{certain} i nesigurne \engl{uncertain}. \textbf{Sigurne} (kategoričke)
informacije smatraju se uvijek istinitima. Primjeri takvih informacije su
matematičke formule ili zdravorazumske činjenice prema kojima je \emph{Zagreb glavni
grad Hrvatske}. Također tu pripadaju i informacije čija je vjerojatnost istinitosti 
izrazito visoka, primjerice: \emph{danas navečer zalazi sunce}. Svaka informacija 
koja nije sigurna je \textbf{nesigurna} informacija. Stupanj sigurnosti informacije
ovisi o kontekstu u kojem se procjenjuje, primjerice: \emph{Hrvatska ima jako dobru poljoprivredu},  
\emph{student Ivo ima 20 godina}, \emph{tipkovnica ima 104 tipke}\dots \@
Postoje još brojne druge dimenzije informacije, kao što
su neodređenost ili vjerojatnost informacije, o kojima je moguće 
više pronaći u literaturi posvećenoj formalnom zaključivanju
\citep{wang2004ontology}. 

\textbf{Subjektivne informacije} ukorjenjene su u mišljenjima, stavovima i vjerovanjima ljudi
koje ih iznose. Odabir studija za studenta često uključuje procjenu 
subjektivnih informacija, kao što su \emph{osobni profesionalni interesi}, 
\emph{težina fakultetskih programa}, \emph{mogućnosti zaposlenja nakon završetka fakulteta}. 
Ukoliko isti izbor gledamo više sa \textbf{objektivnog stajališta}, tada 
ćemo se služiti egzaktnim, provjerenim i pouzdanim informacijama, kao što su:
\emph{90\% ljudi koji završe FER nađu posao u 6 mjeseci nakon diplomiranja}, 
\emph{prosječno trajanje studiranja na FER-u je 6 godina}\dots \@
Sadržaj \textbf{hipotetskih informacija} se pretpostavlja i ne ispituje, već služi 
kao priprema za argumentaciju koja slijedi. Moguće je da 
hipotetska informacija nije točna, niti će ikad biti, ali pretpostavljanjem njene istinitosti
istražuju se posljedice hipotetske izjave. 
Primjerice, \emph{recimo da student Ivo uspješno položi PZUIS, student Ivo
bi tada položio sve predmete na doktorskom studiju i ispunio jedan od preduvjeta završetka studija.}. 
Hipotetski dio argumenta \emph{recimo da student Ivo uspješno položi PZUIS}
služi nam kako bismo pokazali posljedice ostvarivanja istinitosti te informacije,
ali pritom ne ispitujemo istinitost same hipotetske izjave. 

Je li informacija objektivna, subjektivna ili hipotetska, te koliko je njen 
stupanj sigurnosti uvelike ovisi o kontekstu \citep{oren2007subjective}.
U oblikovanju konteksta veliku ulogu ima izvor informacije --- agent.  

\section{Agenti argumentacije}

Agent \engl{agent} je autonoman, proaktivan i nezavistan sustav s određenom
ulogom u argumentaciji~\citep{besnard2008elements}. 
Agenti mogu biti sudionici debate, pripadnici porote, liječnici, odvjetnici, ali
i \textbf{računalni sustavi}. Agenti imaju razne uloge u sklopu argumentacije.  Uloga
odvjetnika može biti dokazivanje krivnje osumnjičenog kroz argumentaciju.
Kvalitetu argumentacije odvjetnika evaluiraju agenti --- članovi porote koji
donose odluku o presudi. S obzirom na broj agenata involviranih u argumentaciju, 
razlikujemo argumentaciju jednog agenta --- \textbf{monološku argumentaciju}
\engl{monological argumentation} i više agenata --- \textbf{dijalošku argumentaciju}
\engl{dialogical argumentation}. 

U \textbf{monološkoj argumentaciji} jedan agent iznosi argumente za i protiv određene tvrdnje. 
Ukoliko student Ivo piše blog na temu: 
\emph{Koji fakultet upisati?}, on 
se može služiti objektivnim informacijama (\emph{90\% ljudi se zapoloslilo nakon što je
završilo moj faks}), subjektivnim informacijama (\emph{profesori na faksu su
ljubazni}) ili hipotetskim informacijama (\emph{da sam upisao drugi faks \dots}).
Izrečene izjave strukturiraju se kroz premise i zaključke te formiraju konačan stav 
prema temi (\emph{student Ivo je upisao ispravan fakultet}).
Neki praktični primjeri monološke argumentacije su eseji, blogovi i kritike. 

\textbf{Dijaloška argumentacija} uključuje više od jedne osobe koje 
argumentiraju za ili protiv zadane teme ili tvrdnje. Agenti mogu imati
međusobno suprotstavljene stavove, međusobno se uvjeravati \engl{persuade}
u ispravnost stavova. Rasprava \emph{Koji faks upisati?} može se odvijati 
putem online foruma gdje student Ivo i ostali studenti 
sveučilišta uvjeravaju gimnazijalce koji fakultet odabrati.  
Naglasak u dijaloškoj arugmentaciji je na
procesu iznošenja argumenata agenata i formiranju konačnih 
stavova agenata. Neki primjeri dijaloške argumentacije su rasprave odvjetnika
u sudnici, rasprave u online diskusijama ili pregovaranje cijene 
automobila.

Bez obzira radi li se o monološkoj ili dijaloškoj argumentaciji, 
agenti analiziraju i vrednuju argumentaciju prema brojnim kriterijima
kao što su 
logička valjanost postupaka zaključivanja, 
kvaliteta subjektivnih argumenata ili
pouzdanost objektivnih argumenata. 
Analizom argumentacije često se bavimo nesvjesno
samim time uvjerava li nas argumentacija da
\emph{upišemo FER zbog argumenata studenta Ive},
\emph{kupimo auto marke BMW zbog njegovih karakteristika}
ili \emph{glasamo za određenu stranku zbog njihovog
karizmatičnog čelnika}. 

% Argumentiranje je verbalna i društvena aktivnost s ciljem jačanja (ili
% osporavanja) diskutabilnog stajališta. Sredstvo argumentiranja su prijedlozi
% ili propozicije \engl{propositions} koji opravdavaju ili opovrgavaju stajalište
% nepristranom sucu s mogućnošću nepristranog racionalnog prosuđivanja
% \citep{rahwan2006representing}. Argumentiranje se često proučava u sklopu
% analize diskursa kao specifičan oblik razgovora \citep{palau2009argumentation}.
% Prema teoriji iz \citep{van2003systematic} argumentiranje je uvijek dio
% dijaloga u kojem jedna strana pokušava uvjeriti drugu u ispravnost svojih
% stajališta. Dijalog ili tekst u kojem autor iznosi svoja stajališta s
% pretpostavkom da očekuje protuargumente, kritike ili sumnje se smatra slobodnim
% \engl{free text}. 

% Teorija argumenta i argumentiranja vrlo je vjerno preslikana u pravnom sustavu.
% Prilikom odlučivanja o nekom diskutabilnoj temi, suprotstavljene teme iznose
% svoje argumente vezane uz temu. Iznesene argumente evaluira racionalan i
% nepristran sudac te donosi odluku kojom se rješava diskutabilnost teme.
% Jednostavan primjer primjene argumentiranja u sudstvu bi bio spor oko
% vlasništva zemlje između susjeda. Svaki od susjeda tvrdi da mu pripada
% vlasništvo nad istim posjedom. Dakle, sudovi izgledaju:
% 
% \begin{itemize} 
%     \item Susjed A tvrdi da mu pripada posjed X.  
%     \item Susjed B
% tvrdi da mu pripada posjed X.  
% \end{itemize}
% 
% Oba suda mogu biti podržana različtim argumentima:
% 
% \begin{itemize} 
%     \item Susjed A tvrdi da mu je posjed X pripadao u zadnjih 10 godina.  
%     \item Susjed B tvrdi da mu je posjed X otkupio o drugog vlasnika od susjeda A.  
% \end{itemize}
% 
% Ispravnost argumentata ocjenjuje sudac (ili porota) nakon čega se oba početna
% suda interpretiraju (dodjeljuje im se logička vrijednost istine ili laži).  

%sada kad sam definirao što je uopće argument (vjerojatno će ići velika
%redukcija ovog napisanog), idem objasniti kako pronaći argument u tekstu s
%primjerima

% \section{Definicija argumenta}
% 
% Argument može imati različite namjene. Izražavanjem \textbf{suda}, što se naziva i \textbf{izjavom} \engl{statement}, pridjeljujemo istinosnu vrijednost danoj izjavi. 
\vspace{2 mm}

\fbox{
 \addtolength{\linewidth}{-2\fboxsep}%
 \addtolength{\linewidth}{-2\fboxrule}%
 \begin{minipage}{13 cm}
  Primjer: Završio sam diplomski studij na FERu.
 \end{minipage}
} 

\vspace{2 mm}
\textbf{Uvjetni sud} \engl{conditional statement} je posebna vrsta izjave, jer razdvaja argumente na \textbf{antecedent} i \textbf{konsekvens}. Najčešće se pojavljuje kao pogodbena složena rečenica u obliku "Ako [antecedent], onda [konsekvens]".
\vspace{2 mm}

\fbox{
 \addtolength{\linewidth}{-2\fboxsep}%
 \addtolength{\linewidth}{-2\fboxrule}%
 \begin{minipage}{13 cm}
  Primjer: Ako sam završio diplomski studij na FERu, onda mogu upisati poslijediplomski studij.
 \end{minipage}
} 

\vspace{2 mm}
Čitava rečenica iz gore navedenog primjera smatra se uvjetnim sudom gdje je "Ako sam završio diplomski studij na FERu" antecedent, a "onda mogu upisati poslijediplomski studij" je konsekvens. Kao i običnom sudu, uvjetnom sudu moguće je pridjeliti istinosnu vrijednost. 

\textbf{Argument} je skup izjava, od kojih je jedna \emph{zaključak}, a ostale su \emph{premise}. Zaključak se donosi na temelju premisa.
\vspace{2 mm}

\fbox{
 \addtolength{\linewidth}{-2\fboxsep}%
 \addtolength{\linewidth}{-2\fboxrule}%
 \begin{minipage}{13 cm}
Primjer: Prema statutu je dozvoljeno upisati poslijediplomski studij samo s prosjekom ocjena višim od 3.5 tijekom diplomskog studija. Završio sam diplomski studij s prosjekom ocjena višim od 3.5, tako da bih mogao upisati poslijediplomski studij.
 \end{minipage}
}
\vspace{2 mm}

Ovdje navodnimo dvije premise, iz čega se izvodi zaključak. Argument se valorizira s obzirom na to koliko dobro premise podupiru zaključak. U ovom jednostavnom primjeru, zaključak ovisi o dvije premise koje obje podupiru zaključak. 

\section{Prepoznavanje argumenata}

Prepoznavanje argumenata u rečenici smatra se teškim problemom, i to ne samo računalna. Bogatstvo jezičnog izražaja, složeni rečenični konstrukti, implicitno zaključivanje ili uporaba ironije su samo neki od faktora koji mogu značajno otežati pronalazak argumenata u tekstu. U \citep{harrellcreating} su navedene smjernice za izdvajanje argumenata iz nestrukturiranih tekstova. Nestrukturirani tekstovi mogu biti osvrti, blogovi, odgovori na internetskim forumima, sudski zapisnici, političke zakonske rasprave \dots
U nestrukturiranim tekstovima premise i zaključci su međusobno isprepleteni, ponekad \emph{skriveni između redaka} ili povezani u istim rečenicama. 
\vspace{1 mm}

\fbox{
 \addtolength{\linewidth}{-2\fboxsep}%
 \addtolength{\linewidth}{-2\fboxrule}%
 \begin{minipage}{13 cm}
Primjer: Treba ukinuti porez na nekretnine jer neće povećati prihod u proračunu.
 \end{minipage}
}
\vspace{2 mm}

U ovom primjeru su premisa (povećanje prihoda u proračunu) i zaključak (ukidanje poreza na nekretnine) navedeni u istoj rečenici, stoga je i ovo primjer argumenta.

S obzirom na vrste riječi u rečenici argumenti se često vežu uz karakteristične uzročne i posljedične priloge, frazeme te često korištene singtagme. Svaki jezik ima svoje specifične karakteristike, stoga je identifikaciju tih ključnih riječi potrebno provesti za svaki jezik. Riječi koje upućuju da slijedi premisa mogu biti:
\begin{center}
\begin{tabular}{c c c c}
jer & zbog & iz razloga što & kako bi  \\
budući da & zato što & unatoč & pretpostavka da
\end{tabular}
\end{center}

Na isti način moguće je pronaći riječi koje se vežu uz zaključak. Identifikacijom tipičnih riječi za argumente može se olakšati posao prepoznavanja argumenata u tekstu. 
Svaki klasifikacijski zadatak često se mora znati nositi s šumovitim podacima. 
Autor teksta može iznijeti tuđi argument s kojim se ne slaže, ne navodeći razloge (premise) \engl{discount}.

\fbox{
 \addtolength{\linewidth}{-2\fboxsep}%
 \addtolength{\linewidth}{-2\fboxrule}%
 \begin{minipage}{13 cm}
Primjer: Mada poneki događaji na financijskoj burzi dionica djeluju nasumično, prema financijskim stručnjacima i njihovim dugogodišnjim istraživanjima velikih korporacija, možemo reći kako je globalno financijsko stanje stabilno. 
 \end{minipage}
}
\vspace{2 mm}

U ovom slučaju imamo glavnu premisu o istraživanju burze, koja je \emph{lažno} devalorizirana nasumičnim događajima. Moguće je ponoviti istu premisu drukčijim riječima \engl{repetition}, izraziti iznimno visoku \engl{assurance} ili nisku pouzdanost \engl{hedge} nekog događaja u obliku argumenata. 

\fbox{
 \addtolength{\linewidth}{-2\fboxsep}%
 \addtolength{\linewidth}{-2\fboxrule}%
 \begin{minipage}{13 cm}
Primjer: Očito je kako je ova generacija studenata bolja od prethodne, jer postiže mnogo bolje rezultate. No, po mom mišljenju, iduća generacija bi mogla biti još bolja.
 \end{minipage}
}
\vspace{2 mm}

Rečenične konstrukcije koje uključuju frazeme kao što su "očito je", "po mom mišljenju" i pokazuju tendencije autora prema nekom argumentu ili se koristi kako bi se čitatelju sugeriralo kako je autor lišen predrasuda. \citep{harrellcreating}. 

\textbf{Entimem} \engl{enthymeme} je argument gdje premisa i/ili zaključak nije eksplicitan \citep{bitzer1959aristotle}. Entimemi se rabe u slučajevima kada autor smatra da nije potrebno eksplicitno naglasiti zaključak.
% 
% \section{Struktura argumenata} Odnosi između argumenata mogu biti različiti. 

\vspace{1 mm}

\fbox{\addtolength{\linewidth}{-2\fboxsep}
\addtolength{\linewidth}{-2\fboxrule}
 \begin{minipage}{13 cm}
Primjer: Farmaceutska istraživanja otkrivaju nove lijekove za suzbijanje
bolesti, a zdravstvo mora biti prioritet svake državne organizacije. Dakle,
vlada bi trebala izdvajati veća sredstva za istraživanja.
 \end{minipage}
}

U ovom primjeru, razlikujemo dvije premise:
\begin{itemize}
\item farmaceutska istraživanja otkrivaju nove lijekove i 
\item zdravstvo mora biti prioritet svake državne organizacije,
\end{itemize} 
koje podupiru jedinstveni zaključak: 
\begin{itemize}
\item vlada bi trebala izdvajati veća sredstva za istraživanja.
\end{itemize}

Ovo je primjer jednog odnosa argumentiranja. Argumentiranje može biti:
\begin{itemize}
\label{itemize:naciniarg}
\item jednostavno argumentiranje (stajalište se podupire jednom tvrdnjom), 
\item višestruko argumentiranje (stajalište se podupire s višestrukim tvrdnjama) i
\item složeno argumentiranje (tvrdnje se međusobno \emph{ulančavaju}) koje može biti:
	\begin{itemize}
	\item koordinirano složeno argumentiranje (\emph{paralelno} podupiranje stajališta)
	\item podređeno složeno argumentiranje (tvrdnje se \emph{serijski} podupiru). 
	\end{itemize}
\end{itemize}

Proučavanjem dijaloga i argumenata omogućilo je strukturiranje argumenata u
grafove \citep{wolf2006coherence}. 


\subsection{Povijest mapiranja argumenata}

Analiza i prikazivanje argumenata u pravnom sustavu se nameće samo po sebi.
Prvi koji se pokušao poslužiti mapiranjem argumenata kako bi opisao pravni
slučaj bio je John H. Wigmore. Doduše, njegove metode mapiranja nikad nisu
doživile svoju praktičnu primjenu na većem broju slučaja. Richard Whately
(1859) se također bavio analizom argumenta.

Kasnije se strukturiranje argumenata počelo više koristiti u računarstvu,
točnije, obradi prirodnog jezika \citep{reed2003argumentation}. 

\subsection{Argumentacijske sheme}


Dvije najkorištenije argumentacijs

% 
% 
% \section{Pretpostavke sustava argumentiranja} Dokazi i argumenti su dva bitno
% različita pojma, pa tako izjave: \emph{P je dokaz da vrijedi T} i \emph{P je
% čvrst argument u korist prihvaćanja teze T} nikako ne mogu biti smatrane
% istovjetnima. Prva izjava pripada području matematička logike
% \engl{mathematical reasoning}. U matematičkoj logici
% \citep{bench2007argumentation}: \begin{itemize} \item ne postoje pojmovi
%     nepotpune ili nesigurne informacije, \item zaključci su konačni, \item
%     kontekst rasuđivanja je strogo definiran i \item sustav odlučivanja nije
%     podložan raspravi te je potpuno objektivan.  \end{itemize} 
% 
% Zaključno, cilj argumenta je \textbf{uvjeravanje} \engl{to persuade}.
