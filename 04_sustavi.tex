Predstavljanje i procesiranje argumenata za radi iz različitih potreba i na
različite načine pa tako razlikujemo četiri najčešća tipa sustava:

\begin{enumerate} 
    \item sustavi za zaključivanje na temelju argumenata
    \engl{formalisms for inference of arguments}, 
    \item sustavi za donošenje
    odluka \engl{argumentation-based decision making}, 
    \item analiza argumenata u diskursu \engl{argumentation based dialogues and legal domain} i 
    \item argumentiranje i učenje \engl{argumentation and learning}.  
\end{enumerate}

Svi nabrojani sustavi implementirani su u brojnim programima što će se navesti
u idućim potpoglavljima.


\section{Zaključivanje na temelju argumenata}

Zaključivanje na temelju argumenata svodi se na dokazivanje 

\section{Sustavi za donošenje argumentiranih odluka}

Sustav za donošenje odluka pomaže korisniku u prikazivanju i strukturiranju
argumenata na temelju čega korisnik donosi odluku. Također, moguća je
interakcija sustava s programskim agentima, gdje sustav za donošenje
argumentiranih odluka predlaže povoljne odluke.

\emph{REACT} je sustav za donošenje odluka u koji je moguće unijeti argumente
koji su potom oblikovani kao objašnjenja. Najčešće se koristi u medicini pri
dijagnozama. \emph{RAG} se bavi procjenom rizika \engl{risk assessment} kroz
pružene argumente. \emph{STAR} je sličan RAGsu. 

Aifdb je jos jedan sustav koji se koristi prilikom argumentacije. 
To je dosta popularan sustav razvijen u Dundee, \hat{S}kotskoj. 
