Analiza argumentacije može biti zahtjevna, kompleksna, posebice
za pojedinca. Analizom složenijih argumentacija bave se skupine ljudi,
koji za to koriste pomoć računala. 
Idejni začetnik povezivanja računala i argumentacije je 
\cite{dung1995acceptability}. Dung je prvi formalizirao sustav koji 
koristi oborivu \engl{defeasible} logiku zaključivanja, jer je smatrao da 
je takva logika prikladnija u argumentaciji prava i medicine
od dotad klasične logike \engl{classical logic}. 
Dungov rad smatra se začećem nove grane umjetne inteligencije ---
\textbf{računalne argumentacije} \engl{computational argumentation}, područje 
posvećeno računalnoj obradi argumentacije. 

Jedan od glavnih ciljeva računalne argumentacije je oblikovanje 
\textbf{argument weba}, velike mreže povezanih 
korisnički izraženih argumenata na Webu. Ostvarivanje argument weba 
dijeli se na \citep{Chris2017-REETAW}:
\begin{enumerate}
    \item predstavljanje argumenatacije u računalu \engl{argument representation},
    \item analizu i vizualizaciju argumentacije \engl{argument visualization and analysis},
    \item podučavanje argumentacije \engl{argument pedagogy},
    \item evaluaciju argumentacije \engl{argument evaluation} i
    \item dubinsku analizu argumentacije \engl{argument mining}. 
\end{enumerate}
U argument webu ne postoje nestrukturirane rasprave, već
svaka rasprava slijedi logičku strukturu te je moguće
lako saznati koji je dokaz \emph{globalnog zatopljenja} ili
\emph{koja istraživanja tvrde da cjepivo nije dobro za ljude}.
U nastavku seminara, reći ćemo nešto više o predstavljanju 
argumentacije u računalu u poglavlju~\ref{chap:aif}, 
o analizi i vizualizaciji argumentacije bit će 
govora u poglavlju~\ref{chap:analysis}, dok će evaluacija argumentacije 
bit će obrađena u poglavlju~\ref{chap:eval}. 
Podučavanje argumentacije nastoji učenike i studente učiti kritičkom 
mišljenju i konstrukciji kvalitetnih argumenata, o čemu neće biti 
govora u ostatku seminara, dok je
dubinska analiza argumentacije najnovije multidisciplinarno područje 
koje pokušava metodama obrade prirodnog jezika \engl{natural language processing}
ekstrahirati argumente iz nestruktirarnog teksta
o čemu ćemo se osvrnuti u zaključku kao
posljednjem, i najtežem, koraku ostvarivanja argument weba.

