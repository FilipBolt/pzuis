Analiza argumentacije može biti zahtjevna, kompleksna, posebice
za pojedinca. Analizom složenijih argumentacija bave se skupine ljudi,
koji za to koriste pomoć računala. 
Korištenje računala za analizu argumentacije započeo je 
\cite{dung1995acceptability}. Dung je prvi formalizirao sustav koji 
koristi oborivu \engl{defeasible} logiku zaključivanja, jer je smatrao da 
je takva logika prikladnija u argumentaciji prava i medicine
od dotad klasične logike \engl{classical logic}. Iz Dungovog rada 
nastat će \textbf{Računalna 
argumentacija} \engl{computational argumentation}, područje 
posvećeno računalnom prikazu argumentacije. 

Jedan od glavnih ciljeva računalne argumentacije je oblikovanje 
\textbf{argument weba}, velike mreže povezanih 
korisnički izraženih argumenata na Webu. Ostvarivanje argument weba 
dijeli se na \citep{Chris2017-REETAW}:
\begin{itemize}
    \item predstavljanje argumenatacije u računalu \engl{argument representation},
    \item analizu argumentacije \engl{argument analysis},
    \item podučavanje argumentacije \engl{argument pedagogy},
    \item vizualizaciju argumentacije \engl{argument navigation},
    \item evaluaciju argumentacije \engl{argument evaluation} i
    \item rudarenje argumentacije \engl{argument mining}. 
\end{itemize}


