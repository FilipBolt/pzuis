Čitav niz alata za argumentaciju, kao što su Araucaria, Rationale ili Carnedeas
koriste vlastite formate za označavanje argumentacije u tekstu. Kako bi se
objedinio način razmjene argumenata između ljudi, ali ponajprije između
računalnih agenata, 2005. nastao je prvi prijedlog za AIF \engl{Argument
Interchange Format} jezikom. Baziran na dobro poznatom RDF jeziku, ideja AIF-a
je bila ponuditi dovoljnu ekspresivnost kojom bi se obuhvatile sve
općeprihvaćene teorije argumentacije. Začetak AIF-a se smatra prvim korakom
prema ostvarivanju Argument Web-a. 

Prije AIF-a bilo je nekoliko pokušaja stvaranja zajedničkog jezika, ponajprije
Araucarin AML \engl{Argument Markup Language}. AML, baziran na XML-u
\engl{Extensible Markup Language} osmišljen je za označavanje i analizu
argumentacije u prirodnom jeziku. Sintaksa AML jezika specificirana je pomoću
DTD \engl{Document Type Definition} strukturalnih ograničenja.  

Standardizacija definiranja računalnog jezika bila je nužna zbog tri glavna razloga:
\begin{enumerate}
    \item razmjena dokumenata kroz različite programske agente (primjerice 
korištenje dokumente kreiranog u Araucarii u Compendiumu i obratno), 
    \item kompatiblinost argumentacijskih shema u različitim programskim agentima
(primjerice, Araucaria koristi Toulminovu shemu, dok Compendium samo poznaje 
Waltonove argumentacijske sheme) te 
    \item potrebe da se automatski procesiraju logičke izjave
\end{enumerate}

\section{Specifikacija AIF-a}

AIF odlikuju 
\begin{enumerate}
    \item sintaksa razumljiva programskim agentima,
    \item eksplicitna semantika,
    \item koncepti i proširenja 
    \item objedinjen apstraktni model koncepata i relacija između
koncepata
\end{enumerate}

\subsection{Koncepti i relacije}

Objekti argumentacije predstavljaju se kao skup čvorova povezanih usmjerenim grafom. 
Neformalno se takav usmjereni graf u kontekstu argumentacije naziva
argumentativnom mrežom \engl{argument network} AN\@. Ne 
postoje nikakva ograničenja na oblik grafa koji može poprimiti AN.

\subsection{Čvorovi}

Razlikujemo dvije osnovne vrste čvorova: informacijske čvorove
\engl{information nodes} I-čvorove i shematske \engl{scheme nodes} S-čvorove.
I-čvorovi predstavljaju sadržaj izjava i čvrsto su povezani s temom
argumentativne rasprave, S-čvorovi predstavljaju primjenu obrazaca u
argumentiranju i smatraju se neovisnim o argumentativnoj raspravi. Postoje tri
osnovna tipa obrasca u argumentiranju: posljedica \engl{inference},
preferiranje \engl{preference} i sukob \engl{conflict}. Primjena sheme
radi se kroz S-čvor koji sukladno vrstama obrazaca u argumentiranju može biti
čvor primjene logičke posljedice \engl{inference application node} (RA-čvor), 
čvor primjene preferiranja \engl{preference application node} (PA-čvor) te
čvor primjene sukoba \engl{conflict application node} (CA-čvor).

\subsection{Bridovi}

Čvorovi su povezani usmjerenim bridovima. Kažemo da brid povezuje čvorove A i B
tako da ide iz početnog čvora A u odredišni čvor B. Razlikujemo shematske i podatkovne
bridove. Početne točke shematskih bridova su S-čvorovi, dok su početne točke 
podatkovnih bridova I-čvorovi. Primjerice, čvorovi A i B su povezani usmjerenim bridom 
$A \rightarrow B$. Ukoliko je čvor A početna točka tipa
S-čvor primjene logičke posljedice RA-čvor onda je čvor B zaključak strukture
čvora A. Čvor B može biti S-čvor ili I-čvor. Ukoliko je početni čvor I-čvor onda
odredišni čvor može biti samo S-čvor. 
Ideja iza toga stoji u principu da nije moguće povezati dvije izjave bez da se
specifira relacija (S-čvor) između izjava. 

\cite{chesnevar2006towards} navode sve moguće kombinacije S-čvorova i I-čvorova
sa bridovima uz pripadajuće semantičko značenje u tablici 1. 

\subsection{Primjer AIF}

% Trebam baš napraviti dijagram gdje je netko I-čvor, S-čvor...
