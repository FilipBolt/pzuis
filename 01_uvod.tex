Okruženje u kojem se riječ može argument pojaviti iznimno je bogato.
Pretraživanjem pojma argument putem najveće enciklopedije na internetu,
Wikipedije, moguće je dobiti sve poveznice na argument u mnogo različitih
domena\footnote{ \url{http://en.wikipedia.org/wiki/Argument_(disambiguation)}}.

Na današnjem webu, Web 2.0,korisnici mogu komunicirati s drugim korisnicima ili
web uslugama \engl{web services}, ali moguća je i interakcija između dvaju web
usluga. Sadržaj koji se prikazuje ograničen je dobro poznatim \emph{HTML}
jezikom \engl{HyperText Markup Language}. 

Argumentima je moguće odgovoriti na pitanja: \emph{Zašto je donesena ta odluka?
Koje su prednosti novog zakona? Koje su mane novih mobitela?} Stoga, poznavanje
argumenta u korist neke teze predstavlja iznimno traženu informaciju u mnogim
područjima ljudskog djelovanja kao što su politika, industrija, zakonodavstvo i
druge. 

Izdvajanjem argumenata iz teksta bavi se područje analiza podataka
\engl{information extraction}.   

Argumenti se često referenciraju jedni na druge, pa je tako moguće jednim
argumentom osporiti ili ojačati drugi. Povezanost argumenata stvara podlogu za
stvaranje mreže argumenata grafičkim putem. Dosadašnji pokušaji strukturiranja
argumenta grafičkim putem rezultirali su izradom stabala argumenata.  
