-- Što je argumentiranje

-- Koliko je popularno u svijetu (reddit i drugi primjeri)

-- U kojem obliku ljudi argumentiraju i kako to uglavnom izgleda (primjene: 
online rasprave, pravna domana, znanstvena domena, kritičko promišljanje i
obrazovanje..)

-- Uvođenje primjera koji će se protezati kroz čitav seminar (pending odluka
što će to točno biti). 

-- Sav taj online resurs predstavlja ogroman neiskoristeni izvor podataka.
Online rasprave. Spomenuti kako su resursi nestrukturirani te kao je to jedan
od primarnih razloga neiskorištenog potencijala argumentacijskih resursa. 
Približiti idealnu Semantičkog weba. 

-- Najava ostatka seminara i sto je u kojem poglavlju

\section{Primjer}

-- U ovom primjeru prvo opisujemo problem koji se rješava. Ideja s ovim
primjerom je završiti na dijagramu iz Rationalea koji onda spaja različite
resurse i donosi zaključak

-- 2. Opisujemo okolnosti problema, koja sva znanja imamo na raspolaganju. 
Npr. na jednoj internet raspravi su rekli X, u znanstvenom članku su otkrili
dokaz Y, javno mnijenje ljudi na online diskusijama je Z

 %Argumentiranje i rasprave su sastavni dio naših zivota. Internetske rasprave 
 %jedan su od najzastupljenijih oblika rasprave danas. Prema istraživanju iz
 %2006. (referenca) 
 %
 %Prema istraživanju iz 2017, Reddit (jedna od najpopularnijih internetskih
 %stranica zasnovana na raspravama je 12. najposjećenija stranica u SAD-u, 25. u
 %cijelom svijetu. [referenca]. 
 %
 %Okruženje u kojem se riječ može argument pojaviti iznimno je bogato.
 %Pretraživanjem pojma argument putem najveće enciklopedije na internetu,
 %Wikipedije, moguće je dobiti sve poveznice na argument u mnogo različitih
 %domena\footnote{\url{http://en.wikipedia.org/wiki/Argument  (disambiguation)}}.
 %
 %Na današnjem webu, Web 2.0,korisnici mogu komunicirati s drugim korisnicima ili
 %web uslugama \engl{web services}, ali moguća je i interakcija između dvaju web
 %usluga. Sadržaj koji se prikazuje ograničen je dobro poznatim \emph{HTML}
 %jezikom \engl{HyperText Markup Language}. 
 %
 %Argumentima je moguće odgovoriti na pitanja: \emph{Zašto je donesena ta odluka?
 %Koje su prednosti novog zakona? Koje su mane novih mobitela?} Stoga, poznavanje
 %argumenta u korist neke teze predstavlja iznimno traženu informaciju u mnogim
 %područjima ljudskog djelovanja kao što su politika, industrija, zakonodavstvo i
 %druge. 
 %
 %Izdvajanjem argumenata iz teksta bavi se područje analiza podataka
 %\engl{information extraction}.   
 %
 %Argumenti se često referenciraju jedni na druge, pa je tako moguće jednim
 %argumentom osporiti ili ojačati drugi. Povezanost argumenata stvara podlogu za
 %stvaranje mreže argumenata grafičkim putem. Dosadašnji pokušaji strukturiranja
 %argumenta grafičkim putem rezultirali su izradom stabala argumenata.  
