Argumentiranje je obrazlaganje
zauzetog stajališta s ciljem uvjeravanja publike \citep{walton1990reasoning}.
Ukoliko ovaj seminar nije dovoljno dobar za položiti predmet
\emph{Predstavljanje znanja u skupovima podataka}, 
mogu se poslužiti argumentima koji će tvrditi suprotno s
ciljem uvjeravanja profesora. Pojavom interneta i 
društvenih mreža, 
argumentiranje je postalo pristupačnije no ikad; Reddit,
jedna od najvećih platformi za rasprave, 
imala je preko 1500 (234 različitih) milijuna posjetitelja
mjesečno u 2017.\ godini, koji 
diskutiraju o preko milijun različitih tematika (subreddita)
\footnote{Brojevi preuzeti 7.1.2018. s \url{https://en.wikipedia.org/wiki/Reddit} i 
\url{http://redditmetrics.com/history}}. 

Argumentacija se javlja u 
raznim oblicima: online rasprave
za ili protiv uvođenja valutne klauzule,
nadmetanje odvjetnika za pobjedu u pravnim slučajevima ili
debata oko originalnosti doprinosa znanstvenog rada.
Neke rasprave, kao online rasprave, nemaju 
strukturu ili pravila, dok 
znanstvene rasprave predstavljalju potpunu suprotnost 
s dobro definiranom strukturom rasprave (motivacija, hipoteza, 
dokaz, eksperimenti, zaključak). 
Strukturiranje rasprava olakšava ulazak novih sudionika
u raspravu, kao i kritičku analizu rasprave. Novi sudionik
strukturirane rasprave lakše će uvidjeti
i ocijeniti koji argumentima nedostaje dokaza
ili tko je počinio grešku u zaključivanju \citep{rieppel1992homology}. 
Kako izvora argumentacije ima mnogo, korištenje računala 
nameće se kao prirodan izbor za pomoć pri analizi. 
Računalna obrada argumentacije razvija se vrlo intenzivno u zadnjih 20 godina
s ciljem strukturiranja rasprava, 
donošenja novih zaključaka 
i evaluacije valjanosti argumentacije. 

% -- Sav taj online resurs predstavlja ogroman neiskoristeni izvor podataka.
% Online rasprave. Spomenuti kako su resursi nestrukturirani te kao je to jedan
% od primarnih razloga neiskorištenog potencijala argumentacijskih resursa. 
% Približiti idealnu Semantičkog weba. 

U ostatku seminarskog rada, objasnit će se pojmovi vezani uz
argument i argumentaciju (poglavlje~\ref{chap:arg}), definirati što je
računalna argumentacija i zašto je potrebna (odjeljak~\ref{chap:rac_arg}).
Nakon uvoda u računalnu argumentaciju, govorit će se o dijelovima
računalne argumentacije: 
predstavljanju argumentacije u računalu (poglavlje~\ref{chap:aif}),
analizi argumentacije (poglavlje~\ref{chap:analysis} i
evaluaciji argumentacije u računalu (poglavlje~\ref{chap:eval}).

% \section{Primjer}
% 
% -- U ovom primjeru prvo opisujemo problem koji se rješava. Ideja s ovim
% primjerom je završiti na dijagramu iz Rationalea koji onda spaja različite
% resurse i donosi zaključak
% 
% -- 2. Opisujemo okolnosti problema, koja sva znanja imamo na raspolaganju. 
% Npr. na jednoj internet raspravi su rekli X, u znanstvenom članku su otkrili
% dokaz Y, javno mnijenje ljudi na online diskusijama je Z

 %Argumentiranje i rasprave su sastavni dio naših zivota. Internetske rasprave 
 %jedan su od najzastupljenijih oblika rasprave danas. Prema istraživanju iz
 %2006. (referenca) 
 %
 %Prema istraživanju iz 2017, Reddit (jedna od najpopularnijih internetskih
 %stranica zasnovana na raspravama je 12. najposjećenija stranica u SAD-u, 25. u
 %cijelom svijetu. [referenca]. 
 %
 %Okruženje u kojem se riječ može argument pojaviti iznimno je bogato.
 %Pretraživanjem pojma argument putem najveće enciklopedije na internetu,
 %Wikipedije, moguće je dobiti sve poveznice na argument u mnogo različitih
 %domena\footnote{\url{http://en.wikipedia.org/wiki/Argument  (disambiguation)}}.
 %
 %Na današnjem webu, Web 2.0,korisnici mogu komunicirati s drugim korisnicima ili
 %web uslugama \engl{web services}, ali moguća je i interakcija između dvaju web
 %usluga. Sadržaj koji se prikazuje ograničen je dobro poznatim \emph{HTML}
 %jezikom \engl{HyperText Markup Language}. 
 %
 %Argumentima je moguće odgovoriti na pitanja: \emph{Zašto je donesena ta odluka?
 %Koje su prednosti novog zakona? Koje su mane novih mobitela?} Stoga, poznavanje
 %argumenta u korist neke teze predstavlja iznimno traženu informaciju u mnogim
 %područjima ljudskog djelovanja kao što su politika, industrija, zakonodavstvo i
 %druge. 
 %
 %Izdvajanjem argumenata iz teksta bavi se područje analiza podataka
 %\engl{information extraction}.   
 %
 %Argumenti se često referenciraju jedni na druge, pa je tako moguće jednim
 %argumentom osporiti ili ojačati drugi. Povezanost argumenata stvara podlogu za
 %stvaranje mreže argumenata grafičkim putem. Dosadašnji pokušaji strukturiranja
 %argumenta grafičkim putem rezultirali su izradom stabala argumenata.  
