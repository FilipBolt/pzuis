Odnosi između argumenata mogu biti različiti. 

\vspace{1 mm}

\fbox{
 \addtolength{\linewidth}{-2\fboxsep}%
 \addtolength{\linewidth}{-2\fboxrule}%
 \begin{minipage}{13 cm}
Primjer: Farmaceutska istraživanja otkrivaju nove lijekove za suzbijanje bolesti, a zdravstvo mora biti prioritet svake državne organizacije. Dakle, vlada bi trebala izdvajati veća sredstva za istraživanja.
 \end{minipage}
}

U ovom primjeru, razlikujemo dvije premise:
\begin{itemize}
\item farmaceutska istraživanja otkrivaju nove lijekove i 
\item zdravstvo mora biti prioritet svake državne organizacije,
\end{itemize} 
koje podupiru jedinstveni zaključak: 
\begin{itemize}
\item vlada bi trebala izdvajati veća sredstva za istraživanja.
\end{itemize}

Ovo je primjer jednog odnosa argumentiranja. Argumentiranje može biti:
\begin{itemize}
\label{itemize:naciniarg}
\item jednostavno argumentiranje (stajalište se podupire jednom tvrdnjom), 
\item višestruko argumentiranje (stajalište se podupire s višestrukim tvrdnjama) i
\item složeno argumentiranje (tvrdnje se međusobno \emph{ulančavaju}) koje može biti:
	\begin{itemize}
	\item koordinirano složeno argumentiranje (\emph{paralelno} podupiranje stajališta)
	\item podređeno složeno argumentiranje (tvrdnje se \emph{serijski} podupiru). 
	\end{itemize}
\end{itemize}

Proučavanjem dijaloga i argumenata omogućilo je strukturiranje argumenata u grafove \citep{wolf2006coherence}. 


\subsection{Povijest mapiranja argumenata}

Analiza i prikazivanje argumenata u pravnom sustavu se nameće samo po sebi. Prvi koji se pokušao poslužiti mapiranjem argumenata kako bi opisao pravni slučaj bio je John H. Wigmore. Doduše, njegove metode mapiranja nikad nisu doživile svoju praktičnu primjenu na većem broju slučaja. Richard Whately (1859) se također bavio analizom argumenta.

Kasnije se strukturiranje argumenata počelo više koristiti u računarstvu, točnije, obradi prirodnog jezika \citep{reed2003argumentation}. 

\subsection{Argumentacijske sheme}


Dvije najkorištenije argumentacijs
