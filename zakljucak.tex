U ovom seminarskom radu definirani su argumenti, argumentacija 
te je opisano kako su predstavljeni
argumenti u računalnim sustavima u sklopu računalne argumentacije. 
Detaljnije je opisana računalna argumentacija te su obrazložene 
njene grane: predstavljanje argumentacije, računalna analiza argumenta 
te evaluacija argumentacije. Pokazano je na primjerima
koje su prednosti korištenja računala pri analizi 
ili evaluaciji argumenata.  
Posljednja grana prema ostvarivanju potpunog 
argument weba je rudarenje argumentacije. 

Rudarenje argumentacije, najmlađa grana računalne argumentacije 
pokušava iz nestrukturiranog teksta
prepoznati dijelove argumenta (premise, zaključke)
i logički povezati argumente. To se 
smatra izuzetno teškim problemom 
u sklopu računalne semantike, jer računalo 
mora \emph{shvatiti} bit i implikacije argumenta. 
Održana su već četiri međunarodna  znanstvena 
skupa \emph{ArgMining} (2014.\ -- 2017.) posvećenih novim metodama
ekstrakcije argumenata iz zapisnika pravnih slučajeva,
eseja ili online rasprava.

Razvoj semantičkog weba, strukturiranje 
online sadržaja kroz ontologije i baze znanja, 
inspiriralo je ekvivalentan pokret 
argument weba među
znanstvenicima koji se bave argumentacijom.
Računalna argumentacija 
aktivno radi na razvoju računalnih sustava
za pomoć ljudima pri 
donošenju odluka s predumišljajem \engl{informed decisions},
a jednog dana --
možda niti čovjek neće uvijek biti potreban.

 
 
