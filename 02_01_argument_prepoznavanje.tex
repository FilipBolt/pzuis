Argument može imati različite namjene. Izražavanjem \textbf{suda}, što se naziva i \textbf{izjavom} \engl{statement}, pridjeljujemo istinosnu vrijednost danoj izjavi. 
\vspace{2 mm}

\fbox{
 \addtolength{\linewidth}{-2\fboxsep}%
 \addtolength{\linewidth}{-2\fboxrule}%
 \begin{minipage}{13 cm}
  Primjer: Završio sam diplomski studij na FERu.
 \end{minipage}
} 

\vspace{2 mm}
\textbf{Uvjetni sud} \engl{conditional statement} je posebna vrsta izjave, jer razdvaja argumente na \textbf{antecedent} i \textbf{konsekvens}. Najčešće se pojavljuje kao pogodbena složena rečenica u obliku "Ako [antecedent], onda [konsekvens]".
\vspace{2 mm}

\fbox{
 \addtolength{\linewidth}{-2\fboxsep}%
 \addtolength{\linewidth}{-2\fboxrule}%
 \begin{minipage}{13 cm}
  Primjer: Ako sam završio diplomski studij na FERu, onda mogu upisati poslijediplomski studij.
 \end{minipage}
} 

\vspace{2 mm}
Čitava rečenica iz gore navedenog primjera smatra se uvjetnim sudom gdje je "Ako sam završio diplomski studij na FERu" antecedent, a "onda mogu upisati poslijediplomski studij" je konsekvens. Kao i običnom sudu, uvjetnom sudu moguće je pridjeliti istinosnu vrijednost. 

\textbf{Argument} je skup izjava, od kojih je jedna \emph{zaključak}, a ostale su \emph{premise}. Zaključak se donosi na temelju premisa.
\vspace{2 mm}

\fbox{
 \addtolength{\linewidth}{-2\fboxsep}%
 \addtolength{\linewidth}{-2\fboxrule}%
 \begin{minipage}{13 cm}
Primjer: Prema statutu je dozvoljeno upisati poslijediplomski studij samo s prosjekom ocjena višim od 3.5 tijekom diplomskog studija. Završio sam diplomski studij s prosjekom ocjena višim od 3.5, tako da bih mogao upisati poslijediplomski studij.
 \end{minipage}
}
\vspace{2 mm}

Ovdje navodnimo dvije premise, iz čega se izvodi zaključak. Argument se valorizira s obzirom na to koliko dobro premise podupiru zaključak. U ovom jednostavnom primjeru, zaključak ovisi o dvije premise koje obje podupiru zaključak. 

\section{Prepoznavanje argumenata}

Prepoznavanje argumenata u rečenici smatra se teškim problemom, i to ne samo računalna. Bogatstvo jezičnog izražaja, složeni rečenični konstrukti, implicitno zaključivanje ili uporaba ironije su samo neki od faktora koji mogu značajno otežati pronalazak argumenata u tekstu. U \citep{harrellcreating} su navedene smjernice za izdvajanje argumenata iz nestrukturiranih tekstova. Nestrukturirani tekstovi mogu biti osvrti, blogovi, odgovori na internetskim forumima, sudski zapisnici, političke zakonske rasprave \dots
U nestrukturiranim tekstovima premise i zaključci su međusobno isprepleteni, ponekad \emph{skriveni između redaka} ili povezani u istim rečenicama. 
\vspace{1 mm}

\fbox{
 \addtolength{\linewidth}{-2\fboxsep}%
 \addtolength{\linewidth}{-2\fboxrule}%
 \begin{minipage}{13 cm}
Primjer: Treba ukinuti porez na nekretnine jer neće povećati prihod u proračunu.
 \end{minipage}
}
\vspace{2 mm}

U ovom primjeru su premisa (povećanje prihoda u proračunu) i zaključak (ukidanje poreza na nekretnine) navedeni u istoj rečenici, stoga je i ovo primjer argumenta.

S obzirom na vrste riječi u rečenici argumenti se često vežu uz karakteristične uzročne i posljedične priloge, frazeme te često korištene singtagme. Svaki jezik ima svoje specifične karakteristike, stoga je identifikaciju tih ključnih riječi potrebno provesti za svaki jezik. Riječi koje upućuju da slijedi premisa mogu biti:
\begin{center}
\begin{tabular}{c c c c}
jer & zbog & iz razloga što & kako bi  \\
budući da & zato što & unatoč & pretpostavka da
\end{tabular}
\end{center}

Na isti način moguće je pronaći riječi koje se vežu uz zaključak. Identifikacijom tipičnih riječi za argumente može se olakšati posao prepoznavanja argumenata u tekstu. 
Svaki klasifikacijski zadatak često se mora znati nositi s šumovitim podacima. 
Autor teksta može iznijeti tuđi argument s kojim se ne slaže, ne navodeći razloge (premise) \engl{discount}.

\fbox{
 \addtolength{\linewidth}{-2\fboxsep}%
 \addtolength{\linewidth}{-2\fboxrule}%
 \begin{minipage}{13 cm}
Primjer: Mada poneki događaji na financijskoj burzi dionica djeluju nasumično, prema financijskim stručnjacima i njihovim dugogodišnjim istraživanjima velikih korporacija, možemo reći kako je globalno financijsko stanje stabilno. 
 \end{minipage}
}
\vspace{2 mm}

U ovom slučaju imamo glavnu premisu o istraživanju burze, koja je \emph{lažno} devalorizirana nasumičnim događajima. Moguće je ponoviti istu premisu drukčijim riječima \engl{repetition}, izraziti iznimno visoku \engl{assurance} ili nisku pouzdanost \engl{hedge} nekog događaja u obliku argumenata. 

\fbox{
 \addtolength{\linewidth}{-2\fboxsep}%
 \addtolength{\linewidth}{-2\fboxrule}%
 \begin{minipage}{13 cm}
Primjer: Očito je kako je ova generacija studenata bolja od prethodne, jer postiže mnogo bolje rezultate. No, po mom mišljenju, iduća generacija bi mogla biti još bolja.
 \end{minipage}
}
\vspace{2 mm}

Rečenične konstrukcije koje uključuju frazeme kao što su "očito je", "po mom mišljenju" i pokazuju tendencije autora prema nekom argumentu ili se koristi kako bi se čitatelju sugeriralo kako je autor lišen predrasuda. \citep{harrellcreating}. 

\textbf{Entimem} \engl{enthymeme} je argument gdje premisa i/ili zaključak nije eksplicitan \citep{bitzer1959aristotle}. Entimemi se rabe u slučajevima kada autor smatra da nije potrebno eksplicitno naglasiti zaključak.